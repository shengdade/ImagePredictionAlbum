
%% bare_conf.tex
%% V1.4b
%% 2015/08/26
%% by Michael Shell
%% See:
%% http://www.michaelshell.org/
%% for current contact information.
%%
%% This is a skeleton file demonstrating the use of IEEEtran.cls
%% (requires IEEEtran.cls version 1.8b or later) with an IEEE
%% conference paper.
%%
%% Support sites:
%% http://www.michaelshell.org/tex/ieeetran/
%% http://www.ctan.org/pkg/ieeetran
%% and
%% http://www.ieee.org/

%%*************************************************************************
%% Legal Notice:
%% This code is offered as-is without any warranty either expressed or
%% implied; without even the implied warranty of MERCHANTABILITY or
%% FITNESS FOR A PARTICULAR PURPOSE! 
%% User assumes all risk.
%% In no event shall the IEEE or any contributor to this code be liable for
%% any damages or losses, including, but not limited to, incidental,
%% consequential, or any other damages, resulting from the use or misuse
%% of any information contained here.
%%
%% All comments are the opinions of their respective authors and are not
%% necessarily endorsed by the IEEE.
%%
%% This work is distributed under the LaTeX Project Public License (LPPL)
%% ( http://www.latex-project.org/ ) version 1.3, and may be freely used,
%% distributed and modified. A copy of the LPPL, version 1.3, is included
%% in the base LaTeX documentation of all distributions of LaTeX released
%% 2003/12/01 or later.
%% Retain all contribution notices and credits.
%% ** Modified files should be clearly indicated as such, including  **
%% ** renaming them and changing author support contact information. **
%%*************************************************************************


% *** Authors should verify (and, if needed, correct) their LaTeX system  ***
% *** with the testflow diagnostic prior to trusting their LaTeX platform ***
% *** with production work. The IEEE's font choices and paper sizes can   ***
% *** trigger bugs that do not appear when using other class files.       ***                          ***
% The testflow support page is at:
% http://www.michaelshell.org/tex/testflow/



\documentclass[conference]{IEEEtran}
% Some Computer Society conferences also require the compsoc mode option,
% but others use the standard conference format.
%
% If IEEEtran.cls has not been installed into the LaTeX system files,
% manually specify the path to it like:
% \documentclass[conference]{../sty/IEEEtran}





% Some very useful LaTeX packages include:
% (uncomment the ones you want to load)


% *** MISC UTILITY PACKAGES ***
%
%\usepackage{ifpdf}
% Heiko Oberdiek's ifpdf.sty is very useful if you need conditional
% compilation based on whether the output is pdf or dvi.
% usage:
% \ifpdf
%   % pdf code
% \else
%   % dvi code
% \fi
% The latest version of ifpdf.sty can be obtained from:
% http://www.ctan.org/pkg/ifpdf
% Also, note that IEEEtran.cls V1.7 and later provides a builtin
% \ifCLASSINFOpdf conditional that works the same way.
% When switching from latex to pdflatex and vice-versa, the compiler may
% have to be run twice to clear warning/error messages.






% *** CITATION PACKAGES ***
%
\usepackage{cite}
% cite.sty was written by Donald Arseneau
% V1.6 and later of IEEEtran pre-defines the format of the cite.sty package
% \cite{} output to follow that of the IEEE. Loading the cite package will
% result in citation numbers being automatically sorted and properly
% "compressed/ranged". e.g., [1], [9], [2], [7], [5], [6] without using
% cite.sty will become [1], [2], [5]--[7], [9] using cite.sty. cite.sty's
% \cite will automatically add leading space, if needed. Use cite.sty's
% noadjust option (cite.sty V3.8 and later) if you want to turn this off
% such as if a citation ever needs to be enclosed in parenthesis.
% cite.sty is already installed on most LaTeX systems. Be sure and use
% version 5.0 (2009-03-20) and later if using hyperref.sty.
% The latest version can be obtained at:
% http://www.ctan.org/pkg/cite
% The documentation is contained in the cite.sty file itself.






% *** GRAPHICS RELATED PACKAGES ***
%
\ifCLASSINFOpdf
  % \usepackage[pdftex]{graphicx}
  \usepackage{graphicx}
  % declare the path(s) where your graphic files are
  % \graphicspath{{../pdf/}{../jpeg/}}
  % and their extensions so you won't have to specify these with
  % every instance of \includegraphics
  % \DeclareGraphicsExtensions{.pdf,.jpeg,.png}
\else
  % or other class option (dvipsone, dvipdf, if not using dvips). graphicx
  % will default to the driver specified in the system graphics.cfg if no
  % driver is specified.
  % \usepackage[dvips]{graphicx}
  % declare the path(s) where your graphic files are
  % \graphicspath{{../eps/}}
  % and their extensions so you won't have to specify these with
  % every instance of \includegraphics
  % \DeclareGraphicsExtensions{.eps}
\fi
% graphicx was written by David Carlisle and Sebastian Rahtz. It is
% required if you want graphics, photos, etc. graphicx.sty is already
% installed on most LaTeX systems. The latest version and documentation
% can be obtained at: 
% http://www.ctan.org/pkg/graphicx
% Another good source of documentation is "Using Imported Graphics in
% LaTeX2e" by Keith Reckdahl which can be found at:
% http://www.ctan.org/pkg/epslatex
%
% latex, and pdflatex in dvi mode, support graphics in encapsulated
% postscript (.eps) format. pdflatex in pdf mode supports graphics
% in .pdf, .jpeg, .png and .mps (metapost) formats. Users should ensure
% that all non-photo figures use a vector format (.eps, .pdf, .mps) and
% not a bitmapped formats (.jpeg, .png). The IEEE frowns on bitmapped formats
% which can result in "jaggedy"/blurry rendering of lines and letters as
% well as large increases in file sizes.
%
% You can find documentation about the pdfTeX application at:
% http://www.tug.org/applications/pdftex


\usepackage{booktabs}


% *** MATH PACKAGES ***
%
%\usepackage{amsmath}
% A popular package from the American Mathematical Society that provides
% many useful and powerful commands for dealing with mathematics.
%
% Note that the amsmath package sets \interdisplaylinepenalty to 10000
% thus preventing page breaks from occurring within multiline equations. Use:
%\interdisplaylinepenalty=2500
% after loading amsmath to restore such page breaks as IEEEtran.cls normally
% does. amsmath.sty is already installed on most LaTeX systems. The latest
% version and documentation can be obtained at:
% http://www.ctan.org/pkg/amsmath





% *** SPECIALIZED LIST PACKAGES ***
%
%\usepackage{algorithmic}
% algorithmic.sty was written by Peter Williams and Rogerio Brito.
% This package provides an algorithmic environment fo describing algorithms.
% You can use the algorithmic environment in-text or within a figure
% environment to provide for a floating algorithm. Do NOT use the algorithm
% floating environment provided by algorithm.sty (by the same authors) or
% algorithm2e.sty (by Christophe Fiorio) as the IEEE does not use dedicated
% algorithm float types and packages that provide these will not provide
% correct IEEE style captions. The latest version and documentation of
% algorithmic.sty can be obtained at:
% http://www.ctan.org/pkg/algorithms
% Also of interest may be the (relatively newer and more customizable)
% algorithmicx.sty package by Szasz Janos:
% http://www.ctan.org/pkg/algorithmicx




% *** ALIGNMENT PACKAGES ***
%
%\usepackage{array}
% Frank Mittelbach's and David Carlisle's array.sty patches and improves
% the standard LaTeX2e array and tabular environments to provide better
% appearance and additional user controls. As the default LaTeX2e table
% generation code is lacking to the point of almost being broken with
% respect to the quality of the end results, all users are strongly
% advised to use an enhanced (at the very least that provided by array.sty)
% set of table tools. array.sty is already installed on most systems. The
% latest version and documentation can be obtained at:
% http://www.ctan.org/pkg/array


% IEEEtran contains the IEEEeqnarray family of commands that can be used to
% generate multiline equations as well as matrices, tables, etc., of high
% quality.




% *** SUBFIGURE PACKAGES ***
%\ifCLASSOPTIONcompsoc
%  \usepackage[caption=false,font=normalsize,labelfont=sf,textfont=sf]{subfig}
%\else
%  \usepackage[caption=false,font=footnotesize]{subfig}
%\fi
% subfig.sty, written by Steven Douglas Cochran, is the modern replacement
% for subfigure.sty, the latter of which is no longer maintained and is
% incompatible with some LaTeX packages including fixltx2e. However,
% subfig.sty requires and automatically loads Axel Sommerfeldt's caption.sty
% which will override IEEEtran.cls' handling of captions and this will result
% in non-IEEE style figure/table captions. To prevent this problem, be sure
% and invoke subfig.sty's "caption=false" package option (available since
% subfig.sty version 1.3, 2005/06/28) as this is will preserve IEEEtran.cls
% handling of captions.
% Note that the Computer Society format requires a larger sans serif font
% than the serif footnote size font used in traditional IEEE formatting
% and thus the need to invoke different subfig.sty package options depending
% on whether compsoc mode has been enabled.
%
% The latest version and documentation of subfig.sty can be obtained at:
% http://www.ctan.org/pkg/subfig




% *** FLOAT PACKAGES ***
%
%\usepackage{fixltx2e}
% fixltx2e, the successor to the earlier fix2col.sty, was written by
% Frank Mittelbach and David Carlisle. This package corrects a few problems
% in the LaTeX2e kernel, the most notable of which is that in current
% LaTeX2e releases, the ordering of single and double column floats is not
% guaranteed to be preserved. Thus, an unpatched LaTeX2e can allow a
% single column figure to be placed prior to an earlier double column
% figure.
% Be aware that LaTeX2e kernels dated 2015 and later have fixltx2e.sty's
% corrections already built into the system in which case a warning will
% be issued if an attempt is made to load fixltx2e.sty as it is no longer
% needed.
% The latest version and documentation can be found at:
% http://www.ctan.org/pkg/fixltx2e


%\usepackage{stfloats}
% stfloats.sty was written by Sigitas Tolusis. This package gives LaTeX2e
% the ability to do double column floats at the bottom of the page as well
% as the top. (e.g., "\begin{figure*}[!b]" is not normally possible in
% LaTeX2e). It also provides a command:
%\fnbelowfloat
% to enable the placement of footnotes below bottom floats (the standard
% LaTeX2e kernel puts them above bottom floats). This is an invasive package
% which rewrites many portions of the LaTeX2e float routines. It may not work
% with other packages that modify the LaTeX2e float routines. The latest
% version and documentation can be obtained at:
% http://www.ctan.org/pkg/stfloats
% Do not use the stfloats baselinefloat ability as the IEEE does not allow
% \baselineskip to stretch. Authors submitting work to the IEEE should note
% that the IEEE rarely uses double column equations and that authors should try
% to avoid such use. Do not be tempted to use the cuted.sty or midfloat.sty
% packages (also by Sigitas Tolusis) as the IEEE does not format its papers in
% such ways.
% Do not attempt to use stfloats with fixltx2e as they are incompatible.
% Instead, use Morten Hogholm'a dblfloatfix which combines the features
% of both fixltx2e and stfloats:
%
% \usepackage{dblfloatfix}
% The latest version can be found at:
% http://www.ctan.org/pkg/dblfloatfix




% *** PDF, URL AND HYPERLINK PACKAGES ***
%
\usepackage{url}
% url.sty was written by Donald Arseneau. It provides better support for
% handling and breaking URLs. url.sty is already installed on most LaTeX
% systems. The latest version and documentation can be obtained at:
% http://www.ctan.org/pkg/url
% Basically, \url{my_url_here}.




% *** Do not adjust lengths that control margins, column widths, etc. ***
% *** Do not use packages that alter fonts (such as pslatex).         ***
% There should be no need to do such things with IEEEtran.cls V1.6 and later.
% (Unless specifically asked to do so by the journal or conference you plan
% to submit to, of course. )


% correct bad hyphenation here
\hyphenation{op-tical net-works semi-conduc-tor}

\makeatletter
\newcommand*{\rom}[1]{\expandafter\@slowromancap\romannumeral #1@}
\makeatother


\begin{document}
%
% paper title
% Titles are generally capitalized except for words such as a, an, and, as,
% at, but, by, for, in, nor, of, on, or, the, to and up, which are usually
% not capitalized unless they are the first or last word of the title.
% Linebreaks \\ can be used within to get better formatting as desired.
% Do not put math or special symbols in the title.
\title{iPre: A Cross-platform Mobile App\\for Image Classification}


% author names and affiliations
% use a multiple column layout for up to three different
% affiliations
\author{\IEEEauthorblockN{Dade Sheng}
\IEEEauthorblockA{Department of Electrical and\\Computer Engineering\\
University of Toronto\\
Email: dade.sheng@mail.utoronto.ca}}

% conference papers do not typically use \thanks and this command
% is locked out in conference mode. If really needed, such as for
% the acknowledgment of grants, issue a \IEEEoverridecommandlockouts
% after \documentclass

% for over three affiliations, or if they all won't fit within the width
% of the page, use this alternative format:
% 
%\author{\IEEEauthorblockN{Michael Shell\IEEEauthorrefmark{1},
%Homer Simpson\IEEEauthorrefmark{2},
%James Kirk\IEEEauthorrefmark{3}, 
%Montgomery Scott\IEEEauthorrefmark{3} and
%Eldon Tyrell\IEEEauthorrefmark{4}}
%\IEEEauthorblockA{\IEEEauthorrefmark{1}School of Electrical and Computer Engineering\\
%Georgia Institute of Technology,
%Atlanta, Georgia 30332--0250\\ Email: see http://www.michaelshell.org/contact.html}
%\IEEEauthorblockA{\IEEEauthorrefmark{2}Twentieth Century Fox, Springfield, USA\\
%Email: homer@thesimpsons.com}
%\IEEEauthorblockA{\IEEEauthorrefmark{3}Starfleet Academy, San Francisco, California 96678-2391\\
%Telephone: (800) 555--1212, Fax: (888) 555--1212}
%\IEEEauthorblockA{\IEEEauthorrefmark{4}Tyrell Inc., 123 Replicant Street, Los Angeles, California 90210--4321}}




% use for special paper notices
%\IEEEspecialpapernotice{(Invited Paper)}




% make the title area
\maketitle

% As a general rule, do not put math, special symbols or citations
% in the abstract
\begin{abstract}
Image Classification has been studied and developed for years. Machine learning models continue to show improvements on classification results. To explore more practical usage of these work, the paper describes the development of a cross-platform mobile app, called iPre, which runs in Android phones and web browsers. Part \rom{1} describes the background, and introduces iPre, which follows a web development pattern and uses services from Amazon Web Services. Part \rom{2} describes the structure of the app, from frontend to backend, with server support, Android development, and classification model. Part \rom{3} demonstrates the app on an Android device. Part \rom{4} demonstrates the app on a Chrome browser. Part \rom{5} concludes the work.
\end{abstract}

% no keywords




% For peer review papers, you can put extra information on the cover
% page as needed:
% \ifCLASSOPTIONpeerreview
% \begin{center} \bfseries EDICS Category: 3-BBND \end{center}
% \fi
%
% For peerreview papers, this IEEEtran command inserts a page break and
% creates the second title. It will be ignored for other modes.
\IEEEpeerreviewmaketitle



\section{Introduction}
% no \IEEEPARstart
% You must have at least 2 lines in the paragraph with the drop letter
% (should never be an issue)
Our brains make vision seem easy. It doesn't take any effort for humans to tell apart a lion and a jaguar, read a sign, or recognize a human's face. But these are actually hard problems to solve with a computer: they only seem easy because our brains are incredibly good at understanding images.

In the last few years the field of machine learning has made tremendous progress on addressing these difficult problems. In particular, it's been found that a kind of model called a deep convolutional neural network can achieve reasonable performance on hard visual recognition tasks -- matching or exceeding human performance in some domains. Basically, a model needs to be trained using existing datasets, after which the model is used to predict new inputs. However, the training process can take months to be complete for quite large training datasets. For predicting an image on a mobile device, it's inappropriate to include training process and let users wait months for results. So it would be a good idea to take advantage of pre-trained neural networks.

Researchers have demonstrated steady progress in computer vision by validating their work against ImageNet -- an academic benchmark for computer vision. Successive models continue to show improvements, each time achieving a new state-of-the-art result: QuocNet, AlexNet, Inception (GoogLeNet), BN-Inception-v2. At the meantime, Python libraries for numerical computation in the field of machine learning have been developed, such as TensorFlow and Theano. Keras is a high-level neural networks API, written in Python and capable of running on top of either TensorFlow or Theano. It was developed with a focus on enabling fast experimentation. There are many trained image classification models for Keras, such as VGG16\cite{vgg16}, VGG19, ResNet50\cite{resnet}, and Inception v3\cite{inception}.

iPre, is a mobile app for image classification using Keras library with VGG16 model. It behaves like a photo album. After registering an account, users can log in and upload their images. All the uploaded images would appear at main page. Clicking one of them, users would see predicted classifications of that image, presented in a pie chart, a bar chart and a probability table. All the images and their predictions are stored permanently, so users can always log back in and browse their images.

\subsection{Web Based Application}
The app is designed by web technologies to make it more flexible. Mobile apps are dependent on platforms. It requires different technologies to design an app in IOS and Android, or other mobile platforms. In some circumstances, a web application stands out because it's running in web browsers of both mobile and PC devices. Therefore, iPre was firstly developed as a web application, and then was migrated to an Android app. The development of Android side is rather straightforward. A web browser was embedded in the Android app to render contents fetched from remote server. In a similar way, it shouldn't take much effort to develop an IOS version, either on iPhone or iPad.

\subsection{Amazon Web Services}
Amazon Web Services (AWS), a subsidiary of Amazon.com, offers a suite of cloud computing services that make up an on-demand computing platform. These services operate from 16 geographical regions across the world. They include Amazon Elastic Compute Cloud, also known as "EC2", and Amazon Simple Storage Service, also known as "S3". 

\subsubsection{S3}
Amazon S3 (Simple Storage Service) is a web service offered by Amazon Web Services. Amazon S3 provides storage through web services interfaces.\cite{wiki:s3} The app stores all user images on S3. Once a user uploaded an image, the image would be temporarily cached on server for classification, and permanently stored in S3. In this way, it provides a stable storage and resource availability. Later on the app can display images by direct links provided by S3.

\subsubsection{EC2}
Amazon Elastic Compute Cloud (EC2) forms a central part of Amazon.com's cloud-computing platform, Amazon Web Services (AWS), by allowing users to rent virtual computers on which to run their own computer applications. EC2 encourages scalable deployment of applications by providing a web service through which a user can boot an Amazon Machine Image (AMI) to configure a virtual machine, which Amazon calls an "instance", containing any software desired.\cite{wiki:ec2} The app uses EC2 instances as backend server, which stores user data and image keys, uploads images to S3, and classify images with Keras library.

\section{Structure}

\subsection{Frontend Design}
For frontend development, it involves web technologies, such as HTML, CSS, and JAVASCRIPT. In order to better fit in mobile devices, responsive web design is employed. Responsive web design (RWD) is an approach to web design aimed at allowing desktop webpages to be viewed in response to the size of the screen or web browser one is viewing with.\cite{wiki:rwd} To achieve better user interface and responsiveness, the app uses a front-end library, called Bootstrap. Bootstrap is a free and open-source front-end web framework for designing websites and web applications. It contains HTML- and CSS-based design templates for typography, forms, buttons, navigation and other interface components, as well as optional JavaScript extensions.\cite{wiki:bootstrap} Following the responsive design pattern, three main pages were designed:

\subsubsection{Login Page}
It contains two parts. First part is registration, where new users can create new accounts. User data would be permanently stored on EC2 instance database. After registration, users can log in with their usernames and passwords. JAVASCRIPT is written here to validate user inputs, to make sure all fields are not empty and passwords are matching certain criterion.

\subsubsection{Main Page}
After successful log-in, users would be taken to this log-in page, where they can view all the images they uploaded and predicted in an album. Also, they can upload new images for classifications. Once new images are uploaded, they would appear in album. There is also an option for logging out accounts. After clicking on a specific image, users would be taken to view pages.

\subsubsection{View Page}
Here shows the classification results of an image. It contains four parts. The first part is the image itself. The remaining parts are different representations of classification results, with each of them having five values, which are organized based on classification probabilities. These includes a pie chart, a bar chart, and a table.

\subsection{Backend Design}

\subsubsection{Flask}
The app is supported by Flask. Flask is a micro web framework written in Python and based on the Werkzeug toolkit and Jinja2 template engine.\cite{wiki:flask} Once a flask app is running on server, it receives requests from clients. It handles all incoming requests, and return data as requested. It works closely with local database which contains user data.

\subsubsection{Celery}
The app takes approximately 7 seconds to classify an image. It would not be user-friendly if it takes too long to respond. The app takes an asynchronous approach. Once an images is uploaded, it starts an asynchronous task using Celery, which executes in the background. Celery is an open source asynchronous task queue or job queue which is based on distributed message passing. While it supports scheduling, its focus is on operations in real time.\cite{wiki:flask}

\subsection{Server Support}
an AWS EC2 instance is serving as a server for the app. The instance type is c4.large. C4 instances are the latest generation of Compute-optimized instances, featuring the highest performing processors and the lowest price/compute performance in EC2. It has 2 virtual central processing units and 3.75GB memory. Several packages are installed and configured:

\subsubsection{Redis-server}
Redis is an open source Database, in-memory data structure store. It communicates with Celery for background task execution.

\subsubsection{Supervisor}
Supervisor is a client/server system that allows its users to monitor and control a number of processes on UNIX-like operating systems. It deals with the management of backend Flask app.

\subsubsection{MySQL}
MySQL is an open-source relational database management system. The app uses a MySQL database to store user information and image keys on S3.

\subsection{Android Development}
With Android Studio as IDE, the Android side of development is straightforward. The app's minimum SDK version is 15 (Android 4.0). Running with full-screen mode, the app is embedded with a WebView, which is standard component in Android to render web contents. With JAVASCRIPT enabled and several functions modified, the web app works more native on Android platform.

\subsection{Classification Model}
As mentioned above, the app uses Keras library with VGG16 model for image classification. To classify an image, firstly, VGG16 model is loaded with weights pre-trained on ImageNet. Secondly, it loads the image with target size 224 by 224. Thirdly, image data is converted to an array and been normalized. Then the image data is passed into classification model and been classified. Finally the app decodes the model outputs as different categories and gives top five 'winners' based on probabilities from high to low.

\section{Android Demonstration}
To better demonstrate the usage of iPre, this part walks through the functionalities from a user's perspective. It's been installed on Nexus 6, an Android device with API 25.

\subsection{Log In}
Figure \ref{fig:login} shows the log-in page, which allows the user register a new account or log in with an existing username. For demo purpose, a user named 'ece1780' has been created. Several images were already uploaded under this account.

\begin{figure}[h!]
  \centering
  \includegraphics[height=\linewidth]{login.png}
  \caption{Log in.}
  \label{fig:login}
\end{figure}

\subsection{Main Page}
Figure \ref{fig:main} shows the main page, including an photo album, which lists all the images of the user 'ece1780'. The user can scroll and browse the images. There is a collapse button at top right corner. After clicking it, the user can choose to upload a new image, or log out current account.

\begin{figure}[h!]
  \centering
  \includegraphics[height=\linewidth]{main.png}
  \caption{Main page.}
  \label{fig:main}
\end{figure}

\subsection{Classification}
If we choose an image from album in main page, iPre would lead us to classification page. Figure \ref{fig:view1} and figure \ref{fig:view2} shows us an example (bottom right corner image from album). There are three presentations of classification results, including a pie chart, a bar chart, and a table. Each presentation shows us five values of probability.

\begin{figure}[h!]
  \centering
  \includegraphics[height=\linewidth]{view1.png}
  \caption{View page.}
  \label{fig:view1}
\end{figure}

\begin{figure}[h!]
  \centering
  \includegraphics[height=\linewidth]{view2.png}
  \caption{View page.}
  \label{fig:view2}
\end{figure}

\section{Browser Demonstration}
Since it's built through a web development approach, it doesn't need to be confined to any specific platform. Figure \ref{fig:pc_login}, \ref{fig:pc_main} and \ref{fig:pc_view} show a web version of the app, which runs on a Chrome browser.

\begin{figure}[h!]
  \centering
  \includegraphics[width=\linewidth]{pc_login.png}
  \caption{Log in.}
  \label{fig:pc_login}
\end{figure}

\begin{figure}[h!]
  \centering
  \includegraphics[width=\linewidth]{pc_main.png}
  \caption{Main page.}
  \label{fig:pc_main}
\end{figure}

\begin{figure}[h!]
  \centering
  \includegraphics[width=\linewidth]{pc_view.png}
  \caption{View page.}
  \label{fig:pc_view}
\end{figure}

% An example of a floating figure using the graphicx package.
% Note that \label must occur AFTER (or within) \caption.
% For figures, \caption should occur after the \includegraphics.
% Note that IEEEtran v1.7 and later has special internal code that
% is designed to preserve the operation of \label within \caption
% even when the captionsoff option is in effect. However, because
% of issues like this, it may be the safest practice to put all your
% \label just after \caption rather than within \caption{}.
%
% Reminder: the "draftcls" or "draftclsnofoot", not "draft", class
% option should be used if it is desired that the figures are to be
% displayed while in draft mode.
%
%\begin{figure}[!t]
%\centering
%\includegraphics[width=2.5in]{myfigure}
% where an .eps filename suffix will be assumed under latex, 
% and a .pdf suffix will be assumed for pdflatex; or what has been declared
% via \DeclareGraphicsExtensions.
%\caption{Simulation results for the network.}
%\label{fig_sim}
%\end{figure}

% Note that the IEEE typically puts floats only at the top, even when this
% results in a large percentage of a column being occupied by floats.


% An example of a double column floating figure using two subfigures.
% (The subfig.sty package must be loaded for this to work.)
% The subfigure \label commands are set within each subfloat command,
% and the \label for the overall figure must come after \caption.
% \hfil is used as a separator to get equal spacing.
% Watch out that the combined width of all the subfigures on a 
% line do not exceed the text width or a line break will occur.
%
%\begin{figure*}[!t]
%\centering
%\subfloat[Case I]{\includegraphics[width=2.5in]{box}%
%\label{fig_first_case}}
%\hfil
%\subfloat[Case II]{\includegraphics[width=2.5in]{box}%
%\label{fig_second_case}}
%\caption{Simulation results for the network.}
%\label{fig_sim}
%\end{figure*}
%
% Note that often IEEE papers with subfigures do not employ subfigure
% captions (using the optional argument to \subfloat[]), but instead will
% reference/describe all of them (a), (b), etc., within the main caption.
% Be aware that for subfig.sty to generate the (a), (b), etc., subfigure
% labels, the optional argument to \subfloat must be present. If a
% subcaption is not desired, just leave its contents blank,
% e.g., \subfloat[].


% An example of a floating table. Note that, for IEEE style tables, the
% \caption command should come BEFORE the table and, given that table
% captions serve much like titles, are usually capitalized except for words
% such as a, an, and, as, at, but, by, for, in, nor, of, on, or, the, to
% and up, which are usually not capitalized unless they are the first or
% last word of the caption. Table text will default to \footnotesize as
% the IEEE normally uses this smaller font for tables.
% The \label must come after \caption as always.
%
%\begin{table}[!t]
%% increase table row spacing, adjust to taste
%\renewcommand{\arraystretch}{1.3}
% if using array.sty, it might be a good idea to tweak the value of
% \extrarowheight as needed to properly center the text within the cells
%\caption{An Example of a Table}
%\label{table_example}
%\centering
%% Some packages, such as MDW tools, offer better commands for making tables
%% than the plain LaTeX2e tabular which is used here.
%\begin{tabular}{|c||c|}
%\hline
%One & Two\\
%\hline
%Three & Four\\
%\hline
%\end{tabular}
%\end{table}


% Note that the IEEE does not put floats in the very first column
% - or typically anywhere on the first page for that matter. Also,
% in-text middle ("here") positioning is typically not used, but it
% is allowed and encouraged for Computer Society conferences (but
% not Computer Society journals). Most IEEE journals/conferences use
% top floats exclusively. 
% Note that, LaTeX2e, unlike IEEE journals/conferences, places
% footnotes above bottom floats. This can be corrected via the
% \fnbelowfloat command of the stfloats package.




\section{Conclusion}
The work is related to Image Classification problem, which is the task of assigning an input image one label from a fixed set of categories. This is one of the core problems in Computer Vision that, despite its simplicity, has a large variety of practical applications. Taking advantage of pre-trained neural networks and high-level libraries, iPre classifies images uploaded to user's albums.
It is platform-independent. Following a web development pattern, iPre was designed to be more flexible. It can be run on any browsers, or migrated to native mobile platforms, such as Android, IOS, and Windows Phone.
It's simple to use and user-friendly. It improves user interface by different classification result interpretations, including two charts and a table.
The app is currently running on an EC2 instance of type c4.large. It takes approximately 7 seconds to classify an image right now. Classification time depends on server performance. The better the server performs, the sooner it classifies. As listed in Table \ref{tab:table1}, classification time depends on EC2 instance types.

\begin{table}[h!]
  \centering
  \caption{performance on difference EC2 instance types.}
  \label{tab:table1}
  \begin{tabular}{cccc}
    \toprule
    Model & vCPU & Mem (GiB) & Time (second)\\
    \midrule
    t2.micro & 1 & 1 & 50.42\\
    t2.small & 1 & 2 & 35.87\\
    m3.medium & 1 & 3.75 & 13.25\\
    c4.large & 2 & 3.75 & 7.61\\
    c4.4xlarge & 16 & 30 & 6.12\\
    \bottomrule
  \end{tabular}
\end{table}




% conference papers do not normally have an appendix


% use section* for acknowledgment
%\section*{Acknowledgment}


%The authors would like to thank...





% trigger a \newpage just before the given reference
% number - used to balance the columns on the last page
% adjust value as needed - may need to be readjusted if
% the document is modified later
%\IEEEtriggeratref{8}
% The "triggered" command can be changed if desired:
%\IEEEtriggercmd{\enlargethispage{-5in}}

% references section

% can use a bibliography generated by BibTeX as a .bbl file
% BibTeX documentation can be easily obtained at:
% http://mirror.ctan.org/biblio/bibtex/contrib/doc/
% The IEEEtran BibTeX style support page is at:
% http://www.michaelshell.org/tex/ieeetran/bibtex/
%\bibliographystyle{IEEEtran}
% argument is your BibTeX string definitions and bibliography database(s)
%\bibliography{IEEEabrv,../bib/paper}
%
% <OR> manually copy in the resultant .bbl file
% set second argument of \begin to the number of references
% (used to reserve space for the reference number labels box)

\begin{thebibliography}{1}

\bibitem{vgg16}
Simonyan, Karen, and Andrew Zisserman. "Very deep convolutional networks for large-scale image recognition." arXiv preprint arXiv:1409.1556 (2014).

\bibitem{resnet}
He, Kaiming, et al. "Deep residual learning for image recognition." Proceedings of the IEEE Conference on Computer Vision and Pattern Recognition. 2016.

\bibitem{inception}
Szegedy, Christian, et al. "Rethinking the inception architecture for computer vision." Proceedings of the IEEE Conference on Computer Vision and Pattern Recognition. 2016.

\bibitem{wiki:s3}
Wikipedia contributors. "S3." Wikipedia, The Free Encyclopedia. Wikipedia, The Free Encyclopedia, 4 Mar. 2017. Web. 30 Mar. 2017.

\bibitem{wiki:ec2}
Wikipedia contributors. "EC2." Wikipedia, The Free Encyclopedia. Wikipedia, The Free Encyclopedia, 28 Apr. 2015. Web. 30 Mar. 2017.

\bibitem{wiki:rwd}
Wikipedia contributors. "Responsive web design." Wikipedia, The Free Encyclopedia. Wikipedia, The Free Encyclopedia, 27 Mar. 2017. Web. 30 Mar. 2017.

\bibitem{wiki:bootstrap}
Wikipedia contributors. "Bootstrap (front-end framework)." Wikipedia, The Free Encyclopedia. Wikipedia, The Free Encyclopedia, 23 Feb. 2017. Web. 30 Mar. 2017.

\bibitem{wiki:flask}
Wikipedia contributors. "Flask (web framework)." Wikipedia, The Free Encyclopedia. Wikipedia, The Free Encyclopedia, 20 Jan. 2017. Web. 30 Mar. 2017.

\bibitem{wiki:celery}
Wikipedia contributors. "Celery (software)." Wikipedia, The Free Encyclopedia. Wikipedia, The Free Encyclopedia, 22 Feb. 2017. Web. 30 Mar. 2017.

\bibitem{wiki:deep}
Chatfield, Ken, Karen Simonyan, Andrea Vedaldi, and Andrew Zisserman. "Return of the devil in the details: Delving deep into convolutional nets." arXiv preprint arXiv:1405.3531 (2014).

\end{thebibliography}


% that's all folks
\end{document}


